
%Author: Siddhesh Wani
%Date: November 27, 2015


%DO NOT 
%  -- change color scheme
%  -- modify version no.

\documentclass[12pt]{article}
\usepackage{tikz}
%\usetikzlibrary{shapes.geometric, arrows}
\usepackage[colorlinks=true, linkcolor=blue, pdfborder={0 0 0}]{hyperref}
\usepackage{standalone}
\usepackage[a4paper]{geometry}
\usepackage{listings}
\usetikzlibrary{arrows,calc,automata,shapes.arrows}

%DO NOT EDIT start
%Define different shapes to be used in flowchart
\tikzstyle{startstop} = [rectangle, rounded corners, minimum width=3cm, minimum height=1cm,text centered, draw=black, fill=red!30]
\tikzstyle{io} = [trapezium, trapezium left angle=70, trapezium right angle=110, minimum width=3cm, minimum height=1cm, text centered, draw=black, fill=blue!30]
\tikzstyle{process} = [rectangle, minimum width=3cm, minimum height=1cm, text centered, draw=black, fill=orange!30]
\tikzstyle{decision} = [diamond, minimum width=3cm, minimum height=1cm, text centered, draw=black, fill=green!30]
\tikzstyle{arrow} = [thick,->,>=stealth]
\tikzstyle{connector} = [signal,draw=black,fill=olive!30]%,text width=1cm,text height=1.5cm,align=center]
%DO NOT EDIT end

\title{Documentation for `Scilab2C' Scilab extension}

\date{}


\begin{document}

\maketitle

\section{Introduction}
\setcounter{section}{1}
`Scilab' is a programming language widely used by researchers and students for scientific computations. Scilab is very easy to use and learn. But with easiness of use comes the 
problem of porting to different platforms (specially embedded devices) and execution speed where C coding takes the precedence. Wouldn't it be nice if there is an utility to 
convert scilab code to C code? Here comes the `Scilab2c' to help. `Scilab2c' is an extension for Scilab for converting scilab files to C code. This document gives an 
introduction about `Scilab2C', how to use it and flowcharts for `Scilab2C' extension.
Contents are arranged as follows:
\begin{itemize}
 \item Need for code conversion
 \item Introduction to `Scilab2C'
 \item How to use `Scilab2C'
 \item Flowcharts for `Scilab2C'.
\end{itemize}
\newpage
\section{Need for code conversion}
As stated in introduction, Scilab is a high level programming language easily understood by majority of the scientific community. Since syntaxing is very easy to learn, 
it is preferred over other programming languages like C, Java or Python. But code written in Scilab is not portable to embedded devices as compiler for compiling scilab code
 to machine languages is not available. On the other hand, code written in C is easily portable to embedded devices because of the avilability of such compilers. But learning C
 and writing code in C is not that easy as syntaxing rules are quite strigent. So, an extension like `Scilab2C' is very useful for converting codes written in scilab to C code.
 Another advantage of C code is that it may consume less space in memory and execute faster than an equivalent scilab code. Hence, a covertor tool like `Scilab2C' is very 
 useful for the user. 
\section{Introduction to `scilab2c'}
\newpage
%\section{How to use `scilab2c'}

%Author: Siddhesh Wani
%Date: December 22, 2015




\documentclass[12pt]{article}
\usepackage{tikz}
\usetikzlibrary{shapes.geometric, arrows}
\usepackage{hyperref}
\usepackage[a4paper]{geometry}
\usepackage[font=small]{caption}


%DO NOT EDIT start
%Define different shapes to be used in flowchart
\tikzstyle{startstop} = [rectangle, rounded corners, minimum width=3cm, minimum height=1cm,text centered, draw=black, fill=red!30]
\tikzstyle{io} = [trapezium, trapezium left angle=70, trapezium right angle=110, minimum width=3cm, minimum height=1cm, text centered, draw=black, fill=blue!30]
\tikzstyle{process} = [rectangle, minimum width=3cm, minimum height=1cm, text centered, draw=black, fill=orange!30]
\tikzstyle{decision} = [diamond, minimum width=3cm, minimum height=1cm, text centered, draw=black, fill=green!30]
\tikzstyle{arrow} = [thick,->,>=stealth]
\tikzstyle{connector} = [signal,draw=black,fill=olive!30]%,text width=1cm,text height=1.5cm,align=center]
%DO NOT EDIT end


\begin{document}



\section{User manual for Scilab2C}



% 
% This flow chart describes steps to be followed for using `scilab2c' extenstion.
% 
% \begin{tikzpicture}[node distance=2cm]
% \label{first}
% 
% \node (start) [startstop] {Start};
% \node (open$_$gui)[process, text width=7cm,  below of=start] {Open scilab2c gui in Scilab by typing `sci2c\_gui' in command window};
% \node (no$_$of$_$files) [decision, below of=open$_$gui, text width=2cm, yshift=-1.5cm] {Does scilab code consists of multiple files?};
% \node (single$_$file) [process, below of=no$_$of$_$files, text width=5cm, yshift=-1cm, xshift=-4cm] {Select scilab file by clicking 'Browse' against 'Main file name'};
% \node (multi$_$files) [process, below of=no$_$of$_$files, text width=5cm, yshift=-1cm, xshift= 4cm] {Select main scilab file by clicking 'Browse' against 'Main file name'};
% \node (sub$_$functions) [process, below of=multi$_$files,text width=6cm]{Select \textbf{folder} containing other scilab functions by clicking 'Browse' against 'Sub-functions'};
% \node (out$_$dir) [process, below of=no$_$of$_$files, text width=5cm, yshift=-5cm] {Select folder in which to store results of conversion};
% \node (run$_$mode) [process, below of=out$_$dir] {Select 'All' for 'Run mode'};
% \node (A) [connector,below of=run$_$mode, signal to=south] {A};
% 
% \draw [arrow] (start) -- (open$_$gui);
% \draw [arrow] (open$_$gui) -- (no$_$of$_$files);
% \draw [arrow] (no$_$of$_$files) -| node[anchor=east] {yes} (multi$_$files);
% \draw [arrow] (no$_$of$_$files) -| node[anchor=west] {no} (single$_$file);
% \draw [arrow] (multi$_$files) -- (sub$_$functions);
% \draw [arrow] (single$_$file) |- (out$_$dir); 
% \draw [arrow] (sub$_$functions) |- (out$_$dir);
% \draw [arrow] (out$_$dir) -- (run$_$mode);
% \draw [arrow] (run$_$mode) -- (A);
% \end{tikzpicture}
% 
% 
% \begin{tikzpicture}[node distance=2cm]
% \node (A) [connector, signal to=north] {A};
% \node (target) [process, below of=A] {Select required 'Target'};
% \node (copy$_$code) [process, below of=target] {Select desired choice for `Copy scilab code into C'};
% \node (tool) [process, below of=copy$_$code] {Select desired choice for `Tool to compile generated C code'};
% \node (convert) [process, below of=tool] {Press `Convert' to start the conversion};
% \node (stop) [startstop, below of=out$_$dir] {Stop};
% 
% \draw [arrow] (A) -- (target); 
% \draw [arrow] (target) -- (copy$_$code);
% \draw [arrow] (copy$_$code) -- (tool);
% \draw [arrow] (tool) -- (convert);
% \draw [arrow] (convert) -- (stop);
% 
% 
% \end{tikzpicture}

This section describes steps to be followed for using Scilab2C. Pre-requisites are mentioned followed by procedure to install Scilab2C.
\\

\subsection{Installation}
\subsubsection{Prerequisites}
There are few prerequisites or some packages must be pre installed before we can use Scilab2C. These are:
\begin{itemize}
  \item scilab $>$=5.5.1.
  \item scilab-arduino toolbox (If using Scilab2C to generate code for Arduino)
  \item Arduino makefile (https://github.com/sudar/Arduino-Makefile). Install using `sudo apt-get install arduino-mk’.
  \item BCM2835 C library for RasberryPi (http://wiringpi.com/)
  \item RasberryPi tools (For cross compiling code for RasberryPi)
\end{itemize}

\subsubsection{Installing Scilab2C}
Before we can use `Scilab2C' extension, we need to install latest version of Scilab2C. Follow following procedure to get latest source code from github repo.
\begin{itemize}
  \item Open terminal window. (Ctrl+Shift+T is shortcut).
  \item Change current directory to `/path/to/scilab/share/scilab/contrib'
  \item Clone the git repo using following command:\\ 
\indent\indent\texttt{git clone https://github.com/siddhu8990/Scilab2C.git}
  \item Make sure a directory named `Scilab2C' is present in `contrib' folder.
  \item Open Scilab.
  \item Run `builder.sce’ file present in ‘Scilab2C/2.3-1’ using 'exec'. This generates binary files from source files.\\
  \indent\indent\texttt{exec(“/path/to/Scilab2C/2.3-1/builder.sce”)}
  \item In `Home/.scilab/scilabx.x.x' make a new file `.scilab' if it does not exist already. Open ‘.scilab’ using suitable editor. Add following line in this file:\\
  \indent\indent\texttt{exec(“/path/to/Scilab2C/2.3-1/loader.sce”)}\\
This will load the `scilab2c' everytime scilab is started.

\end{itemize}

\subsubsection{Installing supporting packages}
Most of the supporting packages or libraries which are required are provided with the toolbox. But they were compiled using
latest source code available at release of toolbox. If you want to use latest libraries, steps to compile the same are listed 
below. You can follow these steps and replace old files with newly generated ones.
\begin{itemize}
 \item scilab-arduino toolbox
  
 %\end{itemize}
  \hypertarget{tools}{}
  \item{\textbf{RasberryPi tools}}
    \begin{itemize}
   \item Make a folder named `RasberryPi tools' somewhere on the harddisk. 
   \item Open terminal and change directory to `RasberryPi tools'. Clone `Tools' repo using `git clone https://github.com/raspberrypi/tools.git'.
   \item Add location of toolchain to your `PATH' variable.\\ \texttt{`export PATH=\$PATH:/location/of/tools/folder/arm-bcm2708/gcc-linaro-arm-linux-gnueabihf-raspbian/bin'} 
  \end{itemize}

   \item{\textbf{ BCM2835 C library for RasberryPi}}
   \begin{itemize}
   \item Before going further, make sure that you have installed `RasberryPi tool' following 
   the instructions given above.
   \item Download latest source code from `\url{http://www.airspayce.com/mikem/bcm2835}'.
   \item Extract source code at some suitable location on harddrive.
   \item Open the terminal window and change current directory to the location where 
   source is extracted.
   \item Execute following command \\
   \texttt{./configure -host=arm CC=arm-linux-gnueabihf-gcc ar=arm-linux-gnueabihf-ar}
   \item Then execute `make' to cross compile the library. Don't do `make install' as it is noramlly next step. 
   \end{itemize}
   
   \item{\textbf{ Cross compiling Lapack and Blas for RasberryPi}}
   \begin{itemize}
    \item Download latest source code for Lapack from `\url{http://www.netlib.org/lapack/}'. Extract source files at some suitable location.
    \item Open file `make.inc.example' given in Lapack folder using some editor.
    \item Edit following items as shown:
    \begin{itemize}
     \item FORTRAN = arm-linux-gnueabihf-gfortran
     \item LOADER = arm-linux-gnueabihf-gfortran
     \item CC = arm-linux-gnueabihf-gcc
     \item ARCH = arm-linux-gnueabihf-ar
     \item RANLIB = arm-linux-gnueabihf-ranlib
    \end{itemize}
    \item Since we are cross compiling for some other platform, normal way compiling 
    will not work.
    \item Open terminal window and change current directory to laplack directory.
    \item We will need to compile BLAS, CBLAS and Lapack separately and in same order.
    \item Change current directory to /path/to/lapack/BLAS/SRC and run `make'. This will generate `librefblas.a' in Lapack folder.
    \item Now change current directory to /path/to/laplack/CBLAS and run `make'. This will generate `libcblas.a' in Lapack folder.
    \item Now change current directory to /path/to/laplack/SRC and run `make'. This will generate `liblapack.a' in Lapack folder.
    Now replace the generated lib files in 'src/c/hardware/rasberrypi/libraries' in `scilab2c' source folder.
   \end{itemize}
   
   \item{\textbf{ GNU Scientific Library (GSL) for RasberryPi}}
   \begin{itemize}
   \item Before going further, make sure that you have installed `RasberryPi tool' following 
   the instructions given \hyperlink{tools}{here}.
   \item Get latest source code for GSL from \url{ftp://ftp.gnu.org/gnu/gsl/}.
   \item Extract source code at some suitable location on harddrive.
   \item Open the terminal window and change current directory to the location where 
   source is extracted.
   \item Execute following command \\
   \texttt{./configure -host=arm CC=arm-linux-gnueabihf-gcc ar=arm-linux-gnueabihf-ar --enable-static}
   \item Then execute `make libgsl.la' to cross compile the library. Don't do `make install' as it is noramlly next step.
   \item Library `libgsl.a' is created in folder `.libs'. By default this folder is hidden.
   \item Now replace the generated lib file in 'src/c/hardware/rasberrypi/libraries' in `scilab2c' source folder.
   
   
   \end{itemize}
   
\end{itemize}

\subsection{Using Scilab2C for C code generation}
Scilab2C extension in scilab can be used for generating C code from a scilab script. Currently it supports three types of target platforms:
\begin{itemize}
 \item \textbf{Standalone C code}: General C code which can be compiled using any compiler.
 \item \textbf{Arduino} : Generated code with few additions can be run on an Arduino board. (A scilab-arduino extension is required)
 \item \textbf{AVR} : Code can be generated for using hardware peripherals of AVR microcontroller

\end{itemize}


You can follow following steps for generating C code using scilab2c extension for required target platforms.
\begin{itemize}
 \item \textbf{Generating standalone C code}
 \begin{enumerate}
  \item Write the scilab script first which is to be converted to C. Scilab code can contain single file or many files, but each file must be a scilab function. 
  There must be one main scilab file in case project contains many files, from which execution of code starts. All scilab files must be in a single folder. 
  
  \begin{figure}[h]
  {
  \centering
   \includegraphics[scale=0.4]{img/sci2cgui}
   \caption{\small{GUI for `Scilab2C'}}
   \label{fig:gui}
  }
  \end{figure}
  
  
  \item Type `sci2c\_gui' or `scilab2c' in scilab console. This will prompt the GUI of Scilab2C extension as shown in figure  \ref{fig:gui}
  
  \begin{figure}
  
  {
  \centering 
   \includegraphics[scale=0.4]{img/gui_pick_file}
   \caption{\small{Select `main' scilab file for conversion}}
   \label{fig:main}
  }
  \end{figure}
  

  \item Click `Browse' next to `Main file name' textbox, browse to location of main scilab file and select it. (Refer figure \ref{fig:main})
  \item If scilab code contains many files, select folder containing these file by clicking `Browse' next to `Sub-functions' textbox.
  \begin{figure}
  \centering
  {
   \includegraphics[scale=0.4]{img/gui_out_pickup}
   \caption {\small{Select output folder}}
   \label{fig:output}
  }
  \end{figure}

  \item Create a new folder somewhere on the disk, preferably in same folder containing scilab files. Select this newly created folder by clicking 
  `Browse' next to `Directory name' textbox. (Refer figure \ref{fig:output})
  \item Choose appropriate options from `Options' box. Different options are explained below:
  \begin{enumerate}
   \item Run mode : If only directory structure is to generated in output directory, select `Generate library'. If only conversion of scilab files is to be done, select
   `Translate'. In case both are to be done, select `All'. 
   \item Target platform : To generate standalone C code, select `Standalone C' from dropdown. (Refer figure \ref{fig:standalone})
  \begin{figure}
  \centering
  {
   \includegraphics[scale=0.4]{img/gui_standalone}
   \caption {\small{Select `Standalone C' from dropdown}}
   \label{fig:standalone}
  }
  \end{figure}

   \item Copy scilab code into C: Select `Yes' or `No' accordingly.
   \item Tool to complie generated C code: Select appropriate option depending upon platform on which generated code will be complied. 
  \end{enumerate}

  \item Confirm everything again and then press `Convert' button. (Refer figure \ref{fig:convert})
  \begin{figure}
  \centering
  {
   \includegraphics[scale=0.4]{img/gui_convert_btn}
   \caption {\small{Select output folder}}
   \label{fig:convert}
  }
  \end{figure}

  \item After clicking `Convert', scilab code will be run in scilab, to check for any errors. If code runs successfully, a prompt will occur asking if you 
  want to continue to code conversion or not. Select `Yes'. If scilab code doesn’t run correctly then code conversion is stopped there itself. 
  Correct the scilab code and follow the steps again.
  \item After selecting `Yes' for code conversion, code conversion starts. If code conversion is done successfully, you will see the message in command window.
  \item Generated code can be seen in output folder. By default a makefile is generated which uses `GCC' compiler to compile the C code. 
  You can compile this code using `make'. Open output folder in terminal and type `make' and press Enter. Once code is compiled successfully, it is run in terminal 
  and output can be seen in terminal window. Check the output for correctness. If code did not behave as expected, correct the scilab code and follow the process again.

 \end{enumerate}

  \item \textbf{Generating code for Arduino}
  \begin{enumerate}
  \item Write the scilab script first which is to be converted to C. Scilab code can contain single file or many files, but each file must be a scilab function. 
  There must be one main scilab file in case project contains many files, from which execution of code starts. All scilab files must be in a single folder. You can verify
  working of scilab script by runnig it on an Arduino board. Modify the script untill code behaves as expected. Once script is finalised, remove the commands 
  `open\_serial' 
  and `close\_serial'.
  
  \item Type `sci2c\_gui' or `scilab2c' in scilab console. This will prompt the GUI of Scilab2C extension as shown in figure  \ref{fig:gui}
  
  \item Click `Browse' next to `Main file name' textbox, browse to location of main scilab file and select it. (Refer figure \ref{fig:main})
  \item If scilab code contains many files, select folder containing these file by clicking `Browse' next to `Sub-functions' textbox.
  
  \item Create a new folder somewhere on the disk, preferably in same folder containing scilab files. Select this newly created folder by clicking 
  `Browse' next to `Directory name' textbox. (Refer figure \ref{fig:output})
  \item Choose appropriate options from `Options' box. Different options are explained below:
  \begin{enumerate}
   \item Run mode : If only directory structure is to generated in output directory, select `Generate library'. If only conversion of scilab files is to be done, select
   `Translate'. In case both are to be done, select `All'. 
   \item Target platform : To generate C code for arduino, select `Arduino' from dropdown. (Refer figure \ref{fig:arduino})
    \begin{figure}
    \centering
    {
    \includegraphics[scale=0.4]{img/gui_arduino}
    \caption {\small{Select `Standalone C' from dropdown}}
    \label{fig:arduino}
    }
    \end{figure}

   \item Copy scilab code into C: Select `Yes' or `No' accordingly.
   \item Tool to complie generated C code: Select appropriate option depending upon platform on which generated code will be complied. 
  \end{enumerate}
  \item Confirm everything again and then press `Convert’ button. (Refer figure \ref{fig:convert})
  \item Code conversion will start, promting different messages in command window. If conversion completes successfully, prompt will occur in command window indicationg the
  same.
  \item Generated code can be seen in output folder. A separate folder named `Arduino' is created, which contains a makefile and an arduino sketch file $-$ sci2c\_arduino.ino.
  \item Open `Makefile' using suitable text editor. Change following parameters according to board and connection:
  \begin{enumerate}
   \item  BOARD\textunderscore TAG
   \item ARDUINO\textunderscore PORT
  \end{enumerate}
  \item Open the terminal and change current directory to the directory containing modified Arduino sketch and then compile by typing `make' in terminal.
  \item If code is compiled successfully, you can upload it to arduino using `make upload' command.
  \item If code doesnot behave as expected, modify sclab code and follow the steps again.

 \end{enumerate}
 \item Generating code for AVR
\end{itemize}





\end{document}
\newpage
%Author: Siddhesh Wani
%Date: March 14, 2016




\documentclass[12pt]{article}
\usepackage{tikz}
\usetikzlibrary{shapes.geometric, arrows}
\usepackage{hyperref}
\usepackage[a4paper]{geometry}
\usepackage[font=small]{caption}
\usepackage{listings}

%DO NOT EDIT start
%Define different shapes to be used in flowchart
\tikzstyle{startstop} = [rectangle, rounded corners, minimum width=3cm, minimum height=1cm,text centered, draw=black, fill=red!30]
\tikzstyle{io} = [trapezium, trapezium left angle=70, trapezium right angle=110, minimum width=3cm, minimum height=1cm, text centered, draw=black, fill=blue!30]
\tikzstyle{process} = [rectangle, minimum width=3cm, minimum height=1cm, text centered, draw=black, fill=orange!30]
\tikzstyle{decision} = [diamond, minimum width=3cm, minimum height=1cm, text centered, draw=black, fill=green!30]
\tikzstyle{arrow} = [thick,->,>=stealth]
\tikzstyle{connector} = [signal,draw=black,fill=olive!30]%,text width=1cm,text height=1.5cm,align=center]
%DO NOT EDIT end


\begin{document}
\newgeometry{right=1.5cm,left=1.5cm}
 \section{Developer Manual for Scilab2C}
 \textbf{Note:} This section is for developers who seek to contribute to this toolbox. Developers who want to use this toolbox, please see 
 \hyperlink{user_manual}{`User manual for Scilab2C'}.\\
 This section explains the structure of the `Scilab2c' toolbox, lists out coding guidelines and instructions for contributing to code.
 
 \subsection{Structure of the toolbox}
 As a developer you will mostly be doing modifications in `macros', `src' and `tests' folder. `macros' folder contains all scilab scripts used 
 for conversion. These macros are grouped into different folders according to their functionality. `src' folder contains 
 all c files which implement scilab functions in c.
 \subsection{Coding guidelines}
 \hypertarget{coding_guide}
 Please follow following guidelines while doing any modifications in the toolbox.
 \begin{enumerate}
  \item Make new folder in `macros' and `src/c' only when new scilab or c file do not fit into any of the existing folders.
  If new folder is added in `macros' make sure you add file `buildmacros.sce' and add that folder in 'etc/scilab2c.start' also.
  \item Add appropriate comments while writing scilab and c functions.
  \item Generated c function names follow following name convention\\
  \texttt{$<$inputs\_type\_\&\_dimensions$><$function\_name$><$outputs\_type\_\&\_dimensions$>$}
  where dimensions can be `0' for scalar types and `2' for vector/matrix types of data. Input and output types can be any one 
  of the specified in table \ref{table:types}\\
  \begin{table}[b]
  \centering 
  \begin{tabular}{|c|c|}
   \hline
   Input/Output data type & Representation \\  \hline
   unsigned 8 bit & u8 \\ \hline
   signed 8 bit & i8 \\ \hline
   unsigned 16 bit & u16 \\ \hline
   signed 16 bit & u16 \\ \hline
   real single precision & s \\ \hline
   real double precision & d \\ \hline
   complex single precision & c \\ \hline
   complex double precision & z \\ \hline
   
  \end{tabular}
  \label{types}
  \end{table}
  \item Interface files provide an interface link to generated c name to a corresponding c function in `src' folder.
  For e.g., `\#define s0coss0(in) scoss(in)' links `s0coss0' with c function `scoss'. `s0coss0' is function name generated for 
  cos function using real single precision data as an input and output is of same type. While `scoss' is implementation 
  of cos in c for scalar inputs of type real single precision.  
  \end{enumerate}

  
 \subsection{Instructions for adding new functions}
 \begin{enumerate}
   \item Decide new scilab function for which support for C conversion is to be added.
   \item Run that function with arguements different data types if function supports different data types. Note down the data types supported.
   \item Go to help file of the same function in Scilab and understand its functionality, cases for input/output arguements (like just one input or variable number of
    inputs etc), dependance of functionality on input arguements etc.
   \item Write C code implementing same functionality as provided by selected scilab function. You may require to write multiple functions (in separate files) as with
   some scilab function functionality is different for different number or types of input arguements.
   \item Name of C function and corresponding c file should correspond to types and number of input/output arguements and name of the function. For naming convention, refer 
   \hyperlink{coding_guide}{'Coding guidelines'}. For example, a function with name `functionname' accepting one input arguement of type `double' scalar and giving output 
   of type `double' scalar, corresponding C function name will be `dfunctionnames'.
   \item In `src' folder in toolbox, find suitable folder which contains other functions providing similar functionality. If you think new function does not fit into any of 
   the current folders than only make new folder. Now in slected folder make a new folder with name as that of c function name. For example, `functionname' in above case.
   Store c files corresponding to this function in this folder. Make sure you have covered all possiblities of input/output arguements.
   \item Write header and interface files for new function. A header file with name `functionname.h' contains definition of all functions defined in that folder. An interface 
   file contains definition linking function name generated during conversion from scilab to c and functions written in c. For more details about interface files, 
   refer \hyperlink{coding_guide}{`Coding guidelines'}.
   \item Store header file and interface file in folders named `includes' and `interfaces' respectively.
   \item Open `includes/sci2clib.h' Add name of the header file here also.
   \item Goto folder `macros/findDeps'. Update files `getAllHeaders.sci', `getAllInterfaces.sci', `getAllSources.sci'. You need to add paths of corresponding files in these files.
   \item Open `macros/ToolInitialisation/INIT\_FillSCI2LibCDirs.sci'. This file basically lists out the functions supported by scilab2c. Whenever a new function is added to
   scilab2c toolbox, this file must be updated. Each function atleast contains the following description. Actual description can contain more lines depending on the 
   function. 
   \begin{lstlisting}[breaklines=true,language=Scilab, numbers=left, numbersep=5pt, numberstyle=\small\color{gray}, frame=single ]
    ClassName = 'Sin';

    // --- Class Annotation. ---
    PrintStringInfo('Adding Class:'+ClassName+'.',GeneralReport,'file','y');
    ClassFileName = fullfile(SCI2CLibCAnnClsDir,ClassName+ExtensionCAnnCls);
    PrintStringInfo('NIN= 1',ClassFileName,'file','y');
    PrintStringInfo('NOUT= 1',ClassFileName,'file','y');
    PrintStringInfo('OUT(1).TP= FA_TP_USER',ClassFileName,'file','y');
    PrintStringInfo('OUT(1).SZ(1)= IN(1).SZ(1)',ClassFileName,'file','y');
    PrintStringInfo('OUT(1).SZ(2)= IN(1).SZ(2)',ClassFileName,'file','y');

    // --- Function List Class. ---
    ClassFileName = fullfile(SCI2CLibCFLClsDir,ClassName+ExtensionCFuncListCls);
    PrintStringInfo('s0'+ArgSeparator+'s0',ClassFileName,'file','y');
    
    // --- Annotation Function And Function List Function. ---
    FunctionName = 'sin'; //BJ : Done AS : Float_Done
    PrintStringInfo('      Adding Function: '+FunctionName+'.',GeneralReport,'file','y');
    INIT_GenAnnFLFunctions(FunctionName,SCI2CLibCAnnFunDir,ClassName,GeneralReport,ExtensionCAnnFun);
    INIT_GenAnnFLFunctions(FunctionName,SCI2CLibCFLFunDir,ClassName,GeneralReport,ExtensionCFuncListFun);

  \end{lstlisting}
  Line 1 specifies the name of class. Single class can describe multiple functions if input/output arguemnts types/numbers
  are same for multiple functions.\\
  Line 3,4 and 5 are same for all functions.\\
  Line 6,7 specify number of inputs and outputs respectively. Line 8 specify type of the output. Line 9 and 10 combinely 
  specify the size of output. Refer to other descriptions for filling out these lines.
  If function supports variable number of inputs, add multiple of these 5 lines for each possibility of number of inputs. 
  For e.g., if function can take 1, 2 or 3 input arguments, there will three sets of these lines.
  Line 14 specifies all possible combinations of inputs/outputs for differnet types and numbers.
  Add as much of these lines as required.
  On line 17, insert correct function name. This is same as scilab function name. Add remaining lines as it is.
  \item Run `builder.sce' and `loader.sce' files in main toolbox folder. 
  \item Now write a scilab script which makes use of newly added function and convert it using scilab2c toolbox. Check the results obtained.
  If results are not as desired, modify c files as required and check again. Do it iteratively untill correct results are obtained.
  \end{enumerate}

 
\end{document}

\newpage
%\section{Flowcharts}
%\newgeometry{top=2cm, bottom=0.5cm,right=1cm,left=1.5cm}
%%Author: Siddhesh Wani
%Date: December 2, 2015




\documentclass[12pt]{article}
\usepackage{tikz}
\usetikzlibrary{shapes.geometric, arrows}
\usepackage{hyperref}
\usepackage[a4paper]{geometry}


%DO NOT EDIT start
%Define different shapes to be used in flowchart
\tikzstyle{startstop} = [rectangle, rounded corners, minimum width=3cm, minimum height=1cm,text centered, draw=black, fill=red!30]
\tikzstyle{io} = [trapezium, trapezium left angle=70, trapezium right angle=110, minimum width=3cm, minimum height=1cm, text centered, draw=black, fill=blue!30]
\tikzstyle{process} = [rectangle, minimum width=3cm, minimum height=1cm, text centered, draw=black, fill=orange!30]
\tikzstyle{decision} = [diamond, minimum width=3cm, minimum height=1cm, text centered, draw=black, fill=green!30]
\tikzstyle{arrow} = [thick,->,>=stealth]
\tikzstyle{connector} = [signal,draw=black,fill=olive!30]%,text width=1cm,text height=1.5cm,align=center]
%DO NOT EDIT end




\begin{document}
\newgeometry{top=2cm, bottom=0.5cm,right=1cm,left=1.5cm}

\vspace*{1cm}
\begin{center}

\section*{scilab2c.sci}
{Siddhesh Wani}
 
\end{center}


{\textbf{Introduction}\\
`scilab2c' is the main function for scilab2c module. Code execution for scilab2c starts from this function. 
`scilab2c' function can be used from command line. It accepts variable number of arguements and depending upon no of arguements
and their values it calls `runsci2c'.\\

\begin{center}
\begin{tikzpicture}[node distance=2cm]

\node (start) [startstop] {Start};
\node (in1) [io, below of=start, text width=4cm] {Accepts any no of inputs between 0 to 6,except 1};
\node (pro1) [process, below of=in1] {Get no of inputs};
\node (no$_$input) [decision, below of=pro1, text width=1cm, yshift=-0.5cm]{Any input present?};
\node (gui) [process, right of=no$_$input, xshift=3cm]{Start scilab2c GUI};
\node (2$_$input) [decision, below of=no$_$input, xshift=-4cm, yshift=-1.5cm, text width=1.5cm]{No of arguements = 2?};
\node (2$_$input$_$pro) [process, right of=2$_$input, xshift=6.2cm, text width=11cm]{Assign:\\ scilab main file = input 1,
folder for other scilab files = []\\ output directory = input 2, runmode = `All' \\buildtool = get buid tool from type of the OS
\\ outformat = `Standalone'};
\node (A) [connector,below of=2$_$input, signal to=south, yshift=-1cm] {A};

\draw [arrow] (start) -- (in1);
\draw [arrow] (in1) -- (pro1);
\draw [arrow] (pro1) -- (no$_$input);
\draw [arrow] (no$_$input) -- node[yshift=0.3cm] {No} (gui);
\draw [arrow] (no$_$input) -| node[west,yshift=0.3cm] {Yes} (2$_$input);
\draw [arrow] (2$_$input) -- node[yshift=0.3cm] {Yes} (2$_$input$_$pro);
\draw [arrow] (2$_$input) -- node[xshift=0.3cm] {No} (A);

\end{tikzpicture}
\end{center}

\begin{tikzpicture}
\node (A) [connector]{A};
\node (3$_$input) [decision, below of=A, xshift=-3cm, yshift=-2.5cm, text width=1.5cm]{No of arguements = 3?};
\node (3$_$input$_$pro) [process, right of=3$_$input, xshift=7cm, text width=9.5cm]{Assign: scilab main file = input 1\\
folder for other scilab files = input 3\\ output directory = input 2, runmode = `All' \\buildtool = get buid tool from type of the OS
\\ outformat = `Standalone'};
\node (4$_$input) [decision, below of=3$_$input, yshift=-3.5cm, text width=1.5cm]{No of arguements = 4?};
\node (4$_$input$_$pro) [process, right of=4$_$input, xshift=7cm, text width=9.5cm]{Assign: scilab main file = input 1\\
folder for other scilab files = input 3\\ output directory = input 2, runmode = input 4 \\buildtool = get buid tool from type of the OS
\\ outformat = `Standalone'};
\node (5$_$input) [decision, below of=4$_$input, yshift=-3.5cm, text width=1.5cm]{No of arguements = 5?};
\node (5$_$input$_$pro) [process, right of=5$_$input, xshift=7cm, text width=9.5cm]{Assign: scilab main file = input 1\\
folder for other scilab files = input 3\\ output directory = input 2, runmode = input 4 \\buildtool = input 5,
outformat = `Standalone'};
\node (6$_$input) [decision, below of=5$_$input, yshift=-3.5cm, text width=1.5cm]{No of arguements = 6?};
\node (6$_$input$_$pro) [process, right of=6$_$input, xshift=7cm, text width=9.5cm]{Assign: scilab main file = input 1\\
folder for other scilab files = input 3\\ output directory = input 2, runmode = input 4 \\buildtool = input 5,
outformat = input 6};
\node (outformat)[decision, below of=6$_$input$_$pro, text width=2cm, yshift=-3.5cm] {Is output format `Standalone'?};
\node (exec) [process, below of=outformat, yshift=-2.5cm]{Execute scilab code};
\node (B) [connector, right of=exec,xshift=2cm]{B};
\node (D) [connector, below of=6$_$input, yshift=-3cm]{D};

\draw [arrow] (A) -| (3$_$input);
\draw [arrow] (3$_$input) -- node[yshift=0.3cm] {Yes} (3$_$input$_$pro);
\draw [arrow] (3$_$input) -- node[xshift=0.3cm] {No} (4$_$input);
\draw [arrow] (4$_$input) -- node[yshift=0.3cm] {Yes} (4$_$input$_$pro);
\draw [arrow] (4$_$input) -- node[xshift=0.3cm] {No} (5$_$input);
\draw [arrow] (5$_$input) -- node[yshift=0.3cm] {Yes} (5$_$input$_$pro);
\draw [arrow] (5$_$input) -- node[xshift=0.3cm] {No} (6$_$input);
\draw [arrow] (6$_$input) -- node[yshift=0.3cm] {Yes} (6$_$input$_$pro);
\draw [arrow] (6$_$input) -- node[xshift=0.3cm] {No} (D);
\draw [arrow] (6$_$input$_$pro) -- (outformat);
\draw [arrow] (outformat) -- node[xshift=0.3cm]{Yes} (exec);
\draw [arrow] (outformat) -| node[yshift=0.3cm]{No} (B.north);
\draw [arrow] (exec.east) --  (B.west);
\draw [arrow] (3$_$input$_$pro.east) --++ (1cm,0) |- (outformat.north);
\draw [arrow] (4$_$input$_$pro.east) --++ (1cm,0);
\draw [arrow] (5$_$input$_$pro.east) --++ (1cm,0);
%\draw [arrow] (6$_$input$_$pro) -- (stop);

\end{tikzpicture}

\begin{tikzpicture}
\node (B) [connector]{B};
\node (D) [connector, left of=B, xshift=-4cm]{D};
\node (mode) [decision, below of=B,text width=2cm, yshift=-2cm]{Is scilab in `std' mode?};
\node (msg) [process, below of=mode, text width=6cm, yshift=-2cm]{Display message dialogue box asking to continue conversion};
\node (continue) [decision, below of=msg,yshift=-3cm, text width=2cm]{User wants to continue conversion?};
\node (conversion) [process, right of=continue, xshift=4cm]{Call `\hyperlink{runsci2c}{runsci2c}'};
\node (stop) [startstop, below of=continue, yshift=-3cm] {Stop};

\draw [arrow] (B) -- (mode);
\draw [arrow] (mode) -- node[xshift=0.3cm]{Yes}(msg);
\draw [arrow] (mode.east) node[yshift=0.3cm, xshift=0.2cm]{No} -- ++(6cm,0) |- (stop.east);
\draw [arrow] (msg) -- (continue);
\draw [arrow] (continue) -- node[yshift=0.3cm]{Yes} (conversion);
\draw [arrow] (continue) -- node[xshift=0.3cm]{No} (stop);
\draw [arrow] (D) |- (stop.west);

\end{tikzpicture}}

\end{document};
%%Author: Siddhesh Wani
%Date: December 8, 2015




\documentclass[12pt]{article}
\usepackage{tikz}
\usetikzlibrary{shapes.geometric, arrows}
\usepackage{hyperref}
\usepackage[a4paper]{geometry}
%DO NOT EDIT start
%Define different shapes to be used in flowchart
\tikzstyle{startstop} = [rectangle, rounded corners, minimum width=3cm, minimum height=1cm,text centered, draw=black, fill=red!30]
\tikzstyle{io} = [trapezium, trapezium left angle=70, trapezium right angle=110, minimum width=3cm, minimum height=1cm, text centered, draw=black, fill=blue!30]
\tikzstyle{process} = [rectangle, minimum width=3cm, minimum height=1cm, text centered, draw=black, fill=orange!30]
\tikzstyle{decision} = [diamond, minimum width=3cm, minimum height=1cm, text centered, draw=black, fill=green!30]
\tikzstyle{arrow} = [thick,->,>=stealth]
\tikzstyle{connector} = [signal,draw=black,fill=olive!30]%,text width=1cm,text height=1.5cm,align=center]
%DO NOT EDIT end




\begin{document}
\newgeometry{top=1cm, bottom=0.5cm,right=1cm,left=1.5cm}

\vspace*{2cm}
\begin{center}

\section*{\hypertarget{runsci2c}runsci2c.sci}
{Siddhesh Wani}\\
 
\end{center}


\textbf{Introduction}\\
'runsci2c' is called by 'scilab2c'. It calls some initialisation modules and then does the conversion. After conversion is complete, it copies the source files in output 
directory and then calls other module to generate makefile for compilation.\\


\begin{center}

\begin{tikzpicture}[node distance=2cm]

\node (start) [startstop] {Start};
\node (input)[io, below of=start, text width=7cm] {Scilab main file, other supporting scilab files, output directory,Run mode, Build tool, Target platform};
\node (initialise) [process, below of=input]{Initialise `ShareInfo'  and  `FileInfo' structures (\hyperlink{initSCI2C}{INIT\_SCI2C})};
\node (getMode) [process, below of=initialise]{Get `RunMode'};
\node (runMode) [decision, below of=getMode, text width=3cm, yshift=-2cm]{Is RunMode All or Generate library structure?};
\node (genLib) [process, right of=runMode, xshift=5cm, text width=5cm]{Initialise library structure (\hyperlink{initGenLib}{INIT\_GenLibraries})};
\node (loadLib) [process, below of=runMode, yshift=-2cm] {Load library information (\hyperlink{initGenLib}{INIT\_LoadLibraries})};
\node (A) [connector,below of=loadLib, signal to=south, yshift=-1cm] {A};

\draw [arrow] (start) -- (input);
\draw [arrow] (input) -- (initialise);
\draw [arrow] (initialise) -- (getMode);
\draw [arrow] (getMode) -- (runMode);
\draw [arrow] (runMode) -- node[yshift=0.3cm]{Yes} (genLib);
\draw [arrow] (runMode) -- node[xshift=0.3cm]{No} (loadLib);
\draw [arrow] (loadLib) -- (A);
\end{tikzpicture}

\begin{tikzpicture}[node distance=2cm]
\node (A) [connector]{A}; 
\node (runMode1) [decision, below of=A, text width=2.5cm, yshift=-2cm]{Is RunMode All or Translate?}; 
\node (main) [process, below of=runMode1, yshift=-1cm]{Get main scilab file};
\node (updInfo)[process, below of=main,text width=4cm, yshift=-0.5cm]{Update its info in `FileInfo' (UpdateSCI2CInfo)};
\node (getAST)[process, below of=updInfo, text width=4cm, yshift=-0.5cm]{Generate AST file (AST\_GetASTFile)};
\node (ast2c)[process, below of=getAST, text width=4cm, yshift=-0.5cm]{Generate C code from AST file (AST2Ccode) };
\node (decJoin)[process, below of=ast2c, text width=4cm, yshift=-1cm]{Join variable declarations and C code (JoinDeclarAndCcode)};
\node (checkFile)[decision, below of=decJoin, text width=3cm, yshift=-2cm]{Is any scilab file left for conversion?};
\node (getFile) [process, left of=ast2c, xshift=-4.5cm, text width=3cm]{Get next scilab file};
\node (copy)[process,right of=main, xshift=5cm, text width=4cm]{Copy source files to output directory};
\node (makefile)[process, below of=copy, text width=4cm, yshift=-1cm]{Generate makefile according to target platform (C\_GenerateMakefile};
\node (sci2cHeader)[process, below of=makefile, text width=5cm, yshift=-1cm]{Generate SCI2C header (C\_GenerateSCI2CHeader)};
\node (stop) [startstop, below of=sci2cHeader] {Stop};

\draw [arrow] (A) -- (runMode1);
\draw [arrow] (runMode1.east) -| node[near start,yshift=0.3cm]{No} (copy.north);
\draw [arrow] (runMode1) -- node[xshift=0.3cm]{Yes} (main);
\draw [arrow] (main) -- (updInfo);
\draw [arrow] (updInfo) -- (getAST);
\draw [arrow] (getAST) -- (ast2c);
\draw [arrow] (ast2c) -- (decJoin);
\draw [arrow] (decJoin) -- (checkFile);
\draw [arrow] (checkFile.west) -| node[near start,yshift=0.3cm]{Yes} (getFile.south);
\draw [arrow] (getFile.north) |- (updInfo);
\draw [arrow] (checkFile.east) node[yshift=0.3cm] {No} --++ (1cm,0) |-  (copy.west);
\draw [arrow] (copy) -- (makefile);
\draw [arrow] (makefile) -- (sci2cHeader);
\draw [arrow] (sci2cHeader) -- (stop);

\end{tikzpicture}

\end{center}

\end{document};
%%Author: Siddhesh Wani
%Date: December 9, 2015




\documentclass[12pt]{article}
\usepackage{tikz}
\usetikzlibrary{shapes.geometric, arrows}
\usepackage{hyperref}
\usepackage[a4paper]{geometry}
%DO NOT EDIT start
%Define different shapes to be used in flowchart
\tikzstyle{startstop} = [rectangle, rounded corners, minimum width=3cm, minimum height=1cm,text centered, draw=black, fill=red!30]
\tikzstyle{io} = [trapezium, trapezium left angle=70, trapezium right angle=110, minimum width=3cm, minimum height=1cm, text centered, draw=black, fill=blue!30]
\tikzstyle{process} = [rectangle, minimum width=3cm, minimum height=1cm, text centered, draw=black, fill=orange!30]
\tikzstyle{decision} = [diamond, minimum width=3cm, minimum height=1cm, text centered, draw=black, fill=green!30]
\tikzstyle{arrow} = [thick,->,>=stealth]
\tikzstyle{connector} = [signal,draw=black,fill=olive!30]%,text width=1cm,text height=1.5cm,align=center]
%DO NOT EDIT end




\begin{document}
\newgeometry{top=1cm, bottom=0.5cm,right=1cm,left=1.5cm}

\vspace*{2cm}
\begin{center}

\section*{\hypertarget{initSCI2C}INIT\_SCI2C.sci}
{Siddhesh Wani}\\
 
\end{center}


\textbf{Introduction}\\
`INIT\_SCI2C' initialises sscilab2c extension using the given input parameters. 'FileInfo' and `SharedInfo' structures are initialised and stored in .dat files. Directory 
structure is created. Other few .dat files required by extension are initialised and saved on disk. \\

\begin{center}
\begin{tikzpicture}[node distance=2cm]
\node (start) [startstop] {Start};
\node (input)[io, below of=start, text width=5cm, yshift=-0.5cm] {Scilab main file, other supporting scilab files, output directory,Run mode, Build tool, Target platform};
\node (sharedinfo)[process, below of=input,text width=4cm, yshift=-0.5cm]{Initialise `SharedInfo' structure. \\ INIT\_GenSharedInfo};
\node (fileinfo)[process, below of=sharedinfo,text width=4cm, yshift=-0.5cm]{Initialise `FileInfo' structure. \\ INIT\_GenFileInfo};
\node (remove)[process, below of=fileinfo,text width=4cm, yshift=-0.5cm]{Remove previous file and directories from output folder \\ INIT\_RemoveDirs};
\node (create)[process, below of=remove,text width=4cm, yshift=-0.5cm]{Create new directories in output folder \\ INIT\_CreateDirs};
\node (global)[process, right of=sharedinfo,text width=4cm,xshift=6cm]{Initialise and save structure for global variables};
\node (funinfo)[process, below of=global,text width=4cm]{Initialise and save `FunInfo' structure};
\node (aststack)[process, below of=funinfo,text width=4cm]{Initialise and save `ASTStack'};
\node (save)[process, below of=aststack,text width=4cm]{Save `SharedInfo' and `FileInfo' structures};
\node (stop) [startstop, below of=save]{Stop};


\draw [arrow] (start) -- (input);
\draw [arrow] (input) -- (sharedinfo);
\draw [arrow] (sharedinfo) -- (fileinfo);
\draw [arrow] (fileinfo) -- (remove);
\draw [arrow] (remove) -- (create);
\draw [arrow] (create.east) --++ (1cm,0) |- (global);
\draw [arrow] (global) -- (funinfo);
\draw [arrow] (funinfo) -- (aststack);
\draw [arrow] (aststack) -- (save);
\draw [arrow] (save) -- (stop);

\end{tikzpicture}
\end{center}

\end{document};
%%Author: Siddhesh Wani
%Date: December 9, 2015




\documentclass[12pt]{article}
\usepackage{tikz}
\usetikzlibrary{shapes.geometric, arrows}
\usepackage{hyperref}
\usepackage[a4paper]{geometry}
%DO NOT EDIT start
%Define different shapes to be used in flowchart
\tikzstyle{startstop} = [rectangle, rounded corners, minimum width=3cm, minimum height=1cm,text centered, draw=black, fill=red!30]
\tikzstyle{io} = [trapezium, trapezium left angle=70, trapezium right angle=110, minimum width=3cm, minimum height=1cm, text centered, draw=black, fill=blue!30]
\tikzstyle{process} = [rectangle, minimum width=3cm, minimum height=1cm, text centered, draw=black, fill=orange!30]
\tikzstyle{decision} = [diamond, minimum width=3cm, minimum height=1cm, text centered, draw=black, fill=green!30]
\tikzstyle{arrow} = [thick,->,>=stealth]
\tikzstyle{connector} = [signal,draw=black,fill=olive!30]%,text width=1cm,text height=1.5cm,align=center]
%DO NOT EDIT end




\begin{document}
\newgeometry{top=1cm, bottom=0.5cm,right=1cm,left=1.5cm}

\vspace*{2cm}
\begin{center}

\section*{\hypertarget{initGenLib}INIT\_GenLibraries.sci}
{Siddhesh Wani}\\
 
\end{center}


\textbf{Introduction}\\
`INIT\_GenLibraries' call `INIT\_FillSCI2CLibCDirs' which generates function annotations for the functions supported by scilab2c extension.\\

\begin{tikzpicture}[node distance=2cm]
\node (start) [startstop] {Start};
\node (input) [io, below of=start]{.dat file containing `FileInfo' structure};
\node (fileinfo)[process, below of=input]{Load `FileInfo' structure};
\node (sharedinfo)[process,below of=fileinfo]{Load `SharedInfo' structure};
\node (gen)[process, below of=sharedinfo, text width=5cm]{Generate function annotations for the functions supported \\ INIT\_FillSCI2CLibCDirs};
\node (stop)[startstop, below of=gen]{Stop};

\draw [arrow] (start) -- (input);
\draw [arrow] (input) -- (fileinfo);
\draw [arrow] (fileinfo) -- (sharedinfo);
\draw [arrow] (sharedinfo) -- (gen);
\draw [arrow] (gen) -- (stop);

\end{tikzpicture}


\end{document};
%%Author: Siddhesh Wani
%Date: December 13, 2015




\documentclass[12pt]{article}
\usepackage{tikz}
\usetikzlibrary{shapes.geometric, arrows}
\usepackage{hyperref}
\usepackage[a4paper]{geometry}
%DO NOT EDIT start
%Define different shapes to be used in flowchart
\tikzstyle{startstop} = [rectangle, rounded corners, minimum width=3cm, minimum height=1cm,text centered, draw=black, fill=red!30]
\tikzstyle{io} = [trapezium, trapezium left angle=70, trapezium right angle=110, minimum width=3cm, minimum height=1cm, text centered, draw=black, fill=blue!30]
\tikzstyle{process} = [rectangle, minimum width=3cm, minimum height=1cm, text centered, draw=black, fill=orange!30]
\tikzstyle{decision} = [diamond, minimum width=3cm, minimum height=1cm, text centered, draw=black, fill=green!30]
\tikzstyle{arrow} = [thick,->,>=stealth]
\tikzstyle{connector} = [signal,draw=black,fill=olive!30]%,text width=1cm,text height=1.5cm,align=center]
%DO NOT EDIT end




\begin{document}
\newgeometry{top=1cm, bottom=0.5cm,right=1cm,left=1.5cm}

\vspace*{2cm}
\begin{center}

\section*{\hypertarget{initLoadLib}INIT\_SciLibraries.sci}
{Siddhesh Wani}\\
 
\end{center}


\textbf{Introduction}\\
`INIT\_LoadLibraries' generates list of available function annotations.\\
\begin{center}
 


\begin{tikzpicture}[node distance=2cm]
\node (start) [startstop] {Start};
\node (input) [io, below of=start]{.dat file containing `FileInfo' structure};
\node (fileinfo)[process, below of=input]{Load `FileInfo' structure};
\node (sharedinfo)[process,below of=fileinfo]{Load `SharedInfo' structure};
\node (list) [process, below of=sharedinfo, text width=5cm]{Make list of functions available for conversion \\ FL\_ExtractFuncList };
\node (save)[process, below of=list]{Save list in a .dat file};
\node (stop) [startstop, below of=save]{Stop};


\draw [arrow] (start) -- (input);
\draw [arrow] (input) -- (fileinfo);
\draw [arrow] (fileinfo) -- (sharedinfo);
\draw [arrow] (sharedinfo) -- (list);
\draw [arrow] (list) -- (save);
\draw [arrow] (save) -- (stop);
\end{tikzpicture}
\end{center}

\end{document};
%%Author: Siddhesh Wani
%Date: January 4, 2016




\documentclass[12pt]{article}
\usepackage{tikz}
\usetikzlibrary{shapes.geometric, arrows}
\usepackage{hyperref}
\usepackage[a4paper]{geometry}
%DO NOT EDIT start
%Define different shapes to be used in flowchart
\tikzstyle{startstop} = [rectangle, rounded corners, minimum width=3cm, minimum height=1cm,text centered, draw=black, fill=red!30]
\tikzstyle{io} = [trapezium, trapezium left angle=70, trapezium right angle=110, minimum width=3cm, minimum height=1cm, text centered, draw=black, fill=blue!30]
\tikzstyle{process} = [rectangle, minimum width=3cm, minimum height=1cm, text centered, draw=black, fill=orange!30]
\tikzstyle{decision} = [diamond, minimum width=3cm, minimum height=1cm, text centered, draw=black, fill=green!30]
\tikzstyle{arrow} = [thick,->,>=stealth]
\tikzstyle{connector} = [signal,draw=black,fill=olive!30]%,text width=1cm,text height=1.5cm,align=center]
%DO NOT EDIT end




\begin{document}
\newgeometry{top=1cm, bottom=0.5cm,right=1cm,left=1.5cm}

\vspace*{2cm}
\begin{center} 


\section*{\hypertarget{updateSCI2CInfo}UpdateSCI2CInfo.sci}
{Siddhesh Wani}\\
 
\end{center}

\textbf{Introduction}\\
This fucntion gets next scilab function to be converted and updates its information in `Fileinfo' structure. 

\begin{center}
\begin{tikzpicture}[node distance=2cm]
\node (start) [startstop, xshift=-4cm] {Start};
\node (input) [io, below of=start, text width=4cm]{.dat file containing `FileInfo' structure};
\node (fileinfo)[process, below of=input, text width=4cm, yshift=-0.2cm]{Load `FileInfo' structure};
\node (sharedinfo)[process,below of=fileinfo, text width=4cm, yshift=-0.2cm]{Load `SharedInfo' structure};
\node (nextfun) [process, below of=sharedinfo, text width=4cm, yshift=-0.2cm]{Get name of next function to be converted};
\node (updfileinfo)[process, below of=nextfun, text width=4cm, yshift=-0.5cm]{Update `FileInfo' structure for next function};
\node (updsharedinfo)[process, below of=updfileinfo, text width=4cm, yshift=-0.5cm]{Update `SharedInfo' structure for next function};
\node (equal)[process, below of=updsharedinfo, text width=5cm, yshift=-0.5cm]{Reset parameters for `Equal' operator (INIT\_SharedInfoEqual)};
\node (converted)[process, below of=equal, text width=5cm, yshift=-0.5cm]{Update list of converted functions (FL\_UpdateConverted)};
\node (dir)[process, right of=input, text width=4cm, xshift=6cm, yshift=-0.5cm]{Create new directory for storing function files};
\node (temp) [process, below of=dir, text width=5cm]{Initialise lists of local and Temporary variables};
\node (precision)[process, below of=temp, text width=5cm, yshift=-0.5cm]{Get default precision in function (FA\_GetDefaultPrecision)};
\node (resize)[process, below of=precision, text width=5.5cm, yshift=-0.7cm]{Determine resize approach in function (FA\_GetResizeApproach)};
\node (save)[process, below of=resize, text width=5cm, yshift=-0.7cm]{Save `FileInfo', `SharedInfo', lists of local and temporary variables and list of converted functions};
\node (stop)[startstop, below of=save, yshift=-0.5cm]{Stop};

\draw [arrow] (start) -- (input);
\draw [arrow] (input) -- (fileinfo);
\draw [arrow] (fileinfo) -- (sharedinfo);
\draw [arrow] (sharedinfo) -- (nextfun);
\draw [arrow] (nextfun) -- (updfileinfo);
\draw [arrow] (updfileinfo) -- (updsharedinfo);
\draw [arrow] (updsharedinfo) -- (equal);
\draw [arrow] (equal) -- (converted);
\draw [arrow] (converted.east) --++(1cm,0) |- (dir.west);
\draw [arrow] (dir) -- (temp);
\draw [arrow] (temp) -- (precision);
\draw [arrow] (precision) -- (resize);
\draw [arrow] (resize) -- (save);
\draw [arrow] (save) -- (stop);





\end{tikzpicture}
\end{center}


\end{document};
%%Author: Siddhesh Wani
%Date: January 7, 2016




\documentclass[12pt]{article}
\usepackage{tikz}
\usetikzlibrary{shapes.geometric, arrows}
\usepackage{hyperref}
\usepackage[a4paper]{geometry}
%DO NOT EDIT start
%Define different shapes to be used in flowchart
\tikzstyle{startstop} = [rectangle, rounded corners, minimum width=3cm, minimum height=1cm,text centered, draw=black, fill=red!30]
\tikzstyle{io} = [trapezium, trapezium left angle=70, trapezium right angle=110, minimum width=3cm, minimum height=1cm, text centered, draw=black, fill=blue!30]
\tikzstyle{process} = [rectangle, minimum width=3cm, minimum height=1cm, text centered, draw=black, fill=orange!30]
\tikzstyle{decision} = [diamond, minimum width=3cm, minimum height=1cm, text centered, draw=black, fill=green!30]
\tikzstyle{arrow} = [thick,->,>=stealth]
\tikzstyle{connector} = [signal,draw=black,fill=olive!30]%,text width=1cm,text height=1.5cm,align=center]
%DO NOT EDIT end




\begin{document}
\newgeometry{top=2cm, bottom=0.5cm,right=1cm,left=1.5cm}

\vspace*{2cm}
\begin{center}

\section*{\hypertarget{AST_GetASTFile}AST\_GetASTFile.sci}
{Siddhesh Wani}\\
 
\end{center}


\textbf{Introduction}\\
Gets next function from `FileInfo' structure and calls `SciFile2ASTFile' with appropriate arguments to generate AST file file from scilab file.

\begin{center}

\begin{tikzpicture}[node distance=2cm]
\node (start) [startstop, xshift=-4cm] {Start};
\node (input) [io, below of=start, text width=6cm]{.dat file containing `FileInfo' structure};
\node (fileinfo)[process, below of=input, text width=6cm, yshift=-0.2cm]{Load `FileInfo' structure};
\node (nextfun) [process, below of=fileinfo, text width=6cm, yshift=-0.2cm]{Get name of next function to be converted};
\node (ast)[process, below of=nextfun, text width=6cm]{Generate AST file from scilab file (SciFile2ASTFile)};
\node (stop)[startstop, below of=ast]{Stop};

\draw [arrow] (start) -- (input);
\draw [arrow] (input) -- (fileinfo);
\draw [arrow] (fileinfo) -- (nextfun);
\draw [arrow] (nextfun) -- (ast);
\draw [arrow] (ast) -- (stop);


\end{tikzpicture}
\end{center}

\end{document};
\end{document}