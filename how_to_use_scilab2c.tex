
%Author: Siddhesh Wani
%Date: November 27, 2015

%Changes to be made while generating new flowchart
%  -- change 'filename.sci' to actual file name
%  -- write some introduction about file -like what it does- in 'Introduction'.
%  -- change author name, obviously :)



%DO NOT 
%  -- change color scheme
%  -- modify version no.

%If any new shapes are added for any particular file, add them here also.
%You can refer 'sample.tex' for various techniques used for drawing flowchart

\documentclass[12pt]{article}
%\documentclass{standalone}
\usepackage{tikz}
\usetikzlibrary{shapes.geometric, arrows}
\usepackage{hyperref}


%DO NOT EDIT start
%Define different shapes to be used in flowchart
\tikzstyle{startstop} = [rectangle, rounded corners, minimum width=3cm, minimum height=1cm,text centered, draw=black, fill=red!30]
\tikzstyle{io} = [trapezium, trapezium left angle=70, trapezium right angle=110, minimum width=3cm, minimum height=1cm, text centered, draw=black, fill=blue!30]
\tikzstyle{process} = [rectangle, minimum width=3cm, minimum height=1cm, text width=5cm, text centered, draw=black, fill=orange!30]
\tikzstyle{decision} = [diamond, minimum width=3cm, minimum height=1cm, text width=4cm, text centered, draw=black, fill=green!30]
\tikzstyle{arrow} = [thick,->,>=stealth]
\tikzstyle{connector} = [signal,draw=black,fill=olive!30]%,text width=1cm,text height=1.5cm,align=center]
%DO NOT EDIT end


\begin{document}



This flow chart describes steps to be followed for using `scilab2c' extenstion.

\begin{tikzpicture}[node distance=2cm]
\label{first}

\node (start) [startstop] {Start};
\node (open$_$gui)[process, text width=7cm,  below of=start] {Open scilab2c gui in Scilab by typing `sci2c\_gui' in command window};
\node (no$_$of$_$files) [decision, below of=open$_$gui, text width=2cm, yshift=-1.5cm] {Does scilab code consists of multiple files?};
\node (single$_$file) [process, below of=no$_$of$_$files, text width=5cm, yshift=-1cm, xshift=-4cm] {Select scilab file by clicking 'Browse' against 'Main file name'};
\node (multi$_$files) [process, below of=no$_$of$_$files, text width=5cm, yshift=-1cm, xshift= 4cm] {Select main scilab file by clicking 'Browse' against 'Main file name'};
\node (sub$_$functions) [process, below of=multi$_$files,text width=6cm]{Select \textbf{folder} containing other scilab functions by clicking 'Browse' against 'Sub-functions'};
\node (out$_$dir) [process, below of=no$_$of$_$files, text width=5cm, yshift=-5cm] {Select folder in which to store results of conversion};
\node (run$_$mode) [process, below of=out$_$dir] {Select 'All' for 'Run mode'};
\node (A) [connector,below of=run$_$mode, signal to=south] {A};

\draw [arrow] (start) -- (open$_$gui);
\draw [arrow] (open$_$gui) -- (no$_$of$_$files);
\draw [arrow] (no$_$of$_$files) -| node[anchor=east] {yes} (multi$_$files);
\draw [arrow] (no$_$of$_$files) -| node[anchor=west] {no} (single$_$file);
\draw [arrow] (multi$_$files) -- (sub$_$functions);
\draw [arrow] (single$_$file) |- (out$_$dir); 
\draw [arrow] (sub$_$functions) |- (out$_$dir);
\draw [arrow] (out$_$dir) -- (run$_$mode);
\draw [arrow] (run$_$mode) -- (A);
\end{tikzpicture}


\begin{tikzpicture}[node distance=2cm]
\node (A) [connector, signal to=north] {A};
\node (target) [process, below of=A] {Select required 'Target'};
\node (copy$_$code) [process, below of=target] {Select desired choice for `Copy scilab code into C'};
\node (tool) [process, below of=copy$_$code] {Select desired choice for `Tool to compile generated C code'};
\node (convert) [process, below of=tool] {Press `Convert' to start the conversion};
\node (stop) [startstop, below of=out$_$dir] {Stop};

\draw [arrow] (A) -- (target); 
\draw [arrow] (target) -- (copy$_$code);
\draw [arrow] (copy$_$code) -- (tool);
\draw [arrow] (tool) -- (convert);
\draw [arrow] (convert) -- (stop);


\end{tikzpicture}


\end{document}